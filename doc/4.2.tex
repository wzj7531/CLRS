\documentclass[a4paper,11pt]{article}
\usepackage{geometry}
\geometry{left=2cm,top=2cm,right=2cm}
\newcommand{\floor}[1]{\left\lfloor #1 \right\rfloor}
\newcommand{\ceiling}[1]{\lceil #1 \rceil}
\usepackage{clrscode3e}
\usepackage{ctex}
\usepackage{amsmath}
\usepackage{forest}
\usepackage{mathtools}

\begin{document}

\section*{4.2 矩阵乘法的Strassen算法}
\begin{codebox}
	\Procname{$\proc{SQUARE-MATRIX-MULTIPLY}(A,B)$}
	\li $\id{n} \gets \attrib{A}{rows}$
	\li let C be a new $n \times n$ matrix
	\li \For $i \gets 1 \To n$
	\li		\Do
				\For $j \gets 1 \To n$
	\li			\Do $c_{ij} \gets 0$
	\li			\For $k \gets 1 \To n$
	\li				\Do	$c_{ij} \gets c_{ij} + a_{ik} \cdot b_{kj}$
						\End
					\End
				\End
	\li	\Return $C$
\end{codebox}
上面过程中,由于三重for循环的每一重都恰好执行n步,而第7行每次执行都花费常量时间,因此花费$\Theta(n^3)$时间。
\paragraph*{矩阵乘法的一个简单的分治算法}
\begin{codebox}
	\Procname{$\proc{SQUARE-MATRIX-MULTIPLY-RECURSIVE}(A,B)$}
	\li $n \gets \attrib{A}{rows}$
	\li Let $C$ be a new $n \times n$ matrix
	\li \If $n \isequal 1$
	\li		\Then $c_{11} \gets a_{11} \cdot b_{11}$
	\li		\Else $patition\ A,\ B,\ and \ C$
	\li			$C_{11} \gets \proc{SQUARE-MATRIX-MULTIPLY-RECURSIVE}(A_{11},B_{11})$\\
	\>\>$+\proc{SQUARE-MATRIX-MULTIPLY-RECURSIVE}(A_{12},B_{21})$
	\li			$C_{12} \gets \proc{SQUARE-MATRIX-MULTIPLY-RECURSIVE}(A_{11},B_{12})$\\
	\>\>$+\proc{SQUARE-MATRIX-MULTIPLY-RECURSIVE}(A_{12},B_{22})$
	\li			$C_{21} \gets	\proc{SQUARE-MATRIX-MULTIPLY-RECURSIVE}(A_{21},B_{11})$\\
	\>\>$+\proc{SQUARE-MATRIX-MULTIPLY-RECURSIVE}(A_{22},B_{21})$
	\li			$C_{22} \gets \proc{SQUARE-MATRIX-MULTIPLY-RECURSIVE}(A_{21},B_{12})$\\
	\>\>$+\proc{SQUARE-MATRIX-MULTIPLY-RECURSIVE}(A_{22},B_{22})$
				\End
	\li	\Return $C$
\end{codebox}
注意,在具体的C实现中,不会真的把$A,B,C$矩阵分为12个新的子矩阵,我们可以计算下标来限定子矩阵的范围,例如,一对坐标表示子矩阵的起始位置,再用一个整数表示该子矩阵的行或者列数,就可以知道子矩阵的范围了,这就是上述伪代码隐藏的实现细节。另外矩阵C在程序开始时初始化为具有A $\times$ B大小的矩阵,每一次递归到基本情况时,就在C上进行计算,待8次递归调用完成后,矩阵C就是A $\times$ B的结果了。\\
现在来分析一下算法的运行时间。由于使用下标计算对$A,B,C$进行子矩阵分解,所以第5行只需$\Theta(1)$的时间,注意,使用下标计算而非通过复制元素(如果这样做则分解矩阵则需要$\Theta(n^2)$的时间)来分解矩阵对总渐进运行时间并无影响。令$T(n)$表示此过程的运行时间。对$n=1$的基本情况,只需要进行一次标量乘法(第4行),因此
\[
	T(n) = \Theta(1)
\]
当$n>1$时是递归情况,第5行花费$\Theta(1)$的时间,第6-9行共8次递归调用,由于每次递归调用完成两个$n/2 \times n/2$矩阵乘法,因此花费时间为$T(n/2)$,8次递归调用总时间为$8T(n/2)$。还需要计算6-9行的4次矩阵加法。每个矩阵包含$n^2/4$个元素,因此每次矩阵加法花费$\Theta(n^2)$时间。由于矩阵加法的的次数是常数,第6-9行进行矩阵假发的总时间为$\Theta(n^2)$(这里仍然使用下标计算的方法讲矩阵加法的结果放置于C的正确位置,由此带来的额外开销为每个元素$\Theta(1)$)。因此,递归情况的总时间为分解时间、递归调用时间以及矩阵加法时间之和:
\[
	T(n) = \Theta(1)+8\Theta(n/2)+\Theta(n^2)=8T(n/2)+\Theta(n^2)
\]
注意,如果通过复制元素来实现矩阵的分解,额外开销为$\Theta(n^2)$,递归式不会发生改变,只是总运行时间将会提高常数倍。综上,
\[
	T(n) = 
	\begin{cases}
	\Theta(1) & \text{若n=1}\\
	8\Theta(n/2)+\Theta(n^2) & \text{若n>1}
	\end{cases}
\]
利用后面学习的主定理,得到的解为$T(n)=\Theta(n^3)$,因此简单的分治算法并不优于暴力算法。注意,因子8决定了递归树中每个节点有几个子节点,进而决定了树的每一层为总和贡献了多少项。如果省略因子8,递归树就变为线性结构,而不是“茂盛的”了,树的每一层只为总和贡献了一项。
\section*{Strassen方法}
Strassen方法的核心思想是用常数次的矩阵加法的代价减少一次矩阵乘法。其递归式如下所示:
\[
	T(n) = 
	\begin{cases}
		\Theta(1) & \text{若n=1} \\
		7\Theta(n/2) + \Theta(n^2) &\text{若n>1}
	\end{cases}
\]
再次利用后面的主定理,$T(n)=\Theta(n^{lg7})$。
\section*{练习}
\paragraph*{4.2-1}
使用Strassen算法计算如下矩阵乘法:
\[
	\begin{gathered}
		\begin{bmatrix}
			1 & 3\\
			7 & 5
		\end{bmatrix}
		\quad
		\begin{bmatrix}
			6 & 8\\
			4 & 2
		\end{bmatrix}
	\end{gathered}
\]
给出计算过程。\\
\paragraph{解:}我们从步骤2开始计算$S_1,S_2,\cdots,S_{10}$:
\begin{align*}
	S_1 = B_{12}-B_{22} &=
	\begin{bmatrix}
	8
	\end{bmatrix}
	-
	\begin{bmatrix}
	2
	\end{bmatrix}
	=
	\begin{bmatrix}
	6
	\end{bmatrix}\\
	S_2 = A_{11}+A_{12} &=
	\begin{bmatrix}
	1
	\end{bmatrix}
	+
	\begin{bmatrix}
	3
	\end{bmatrix}
	=
	\begin{bmatrix}
	4
	\end{bmatrix} \\
	S_3 = A_{21}+A_{22} &=
	\begin{bmatrix}
	7
	\end{bmatrix}
	+
	\begin{bmatrix}
	5
	\end{bmatrix}
	=
	\begin{bmatrix}
	12
	\end{bmatrix} \\	
	S_4 = B_{21}+B_{11} &=
	\begin{bmatrix}
	4
	\end{bmatrix}
	-
	\begin{bmatrix}
	6
	\end{bmatrix}
	=
	\begin{bmatrix}
	-2
	\end{bmatrix} \\	
	S_5 = A_{11}+A_{22} &=
	\begin{bmatrix}
	1
	\end{bmatrix}
	+
	\begin{bmatrix}
	5
	\end{bmatrix}
	=
	\begin{bmatrix}
	6
	\end{bmatrix} \\	
	S_6 = B_{11}+B_{22} &=
	\begin{bmatrix}
	6
	\end{bmatrix}
	+
	\begin{bmatrix}
	2
	\end{bmatrix}
	=
	\begin{bmatrix}
	8
	\end{bmatrix} \\	
	S_7 = A_{12}-A_{22} &=
	\begin{bmatrix}
	3
	\end{bmatrix}
	-
	\begin{bmatrix}
	5
	\end{bmatrix}
	=
	\begin{bmatrix}
	-2
	\end{bmatrix} \\	
	S_8 = B_{21}+B_{22} &=
	\begin{bmatrix}
	4
	\end{bmatrix}
	+
	\begin{bmatrix}
	2
	\end{bmatrix}
	=
	\begin{bmatrix}
	6
	\end{bmatrix} \\	
	S_9 = A_{11}-A_{21} &=
	\begin{bmatrix}
	1
	\end{bmatrix}
	-
	\begin{bmatrix}
	7
	\end{bmatrix}
	=
	\begin{bmatrix}
	-6
	\end{bmatrix} \\	
	S_{10} = B_{11}+B_{12} &=
	\begin{bmatrix}
	6
	\end{bmatrix}
	+
	\begin{bmatrix}
	8
	\end{bmatrix}
	=
	\begin{bmatrix}
	14
	\end{bmatrix}
\end{align*}
然后进行步骤3的计算:
\begin{align*}
	P_{1} = A_{11} \cdot S_1 &=
	\begin{bmatrix}
		6
	\end{bmatrix}\\
	P_{2} = S_{2} \cdot B_{22} &=
	\begin{bmatrix}
		8
	\end{bmatrix}\\
	P_{3} =S_{3} \cdot B_{11} &=
	\begin{bmatrix}
		72
	\end{bmatrix}\\
	P_{4} =A_{22} \cdot S_{4} &=
	\begin{bmatrix}
		-10
	\end{bmatrix}\\
	P_{5} =S_{5} \cdot S_{6} &=
	\begin{bmatrix}
		48
	\end{bmatrix}\\
	P_{6} = S_{7} \cdot S_{8} &=
	\begin{bmatrix}
		-12
	\end{bmatrix}\\
	P_{7} = S_{9} \cdot S_{10} &=
	\begin{bmatrix}
		-84
	\end{bmatrix}
\end{align*}
然后进行步骤4的计算:
\begin{align*}
	C_{11} = P_{5} + P_{4} - P_{2} + P{6} &=
	\begin{bmatrix}
		18
	\end{bmatrix}\\
	C_{12} = P_{1} + P_{2} &=
	\begin{bmatrix}
		14
	\end{bmatrix}\\
	C_{21} = P_{3} + P_{4} &=
	\begin{bmatrix}
		62
	\end{bmatrix}\\
	C_{22} = P_{5}+P{1}-P_{3}-P_{7} &=
	\begin{bmatrix}
		66
	\end{bmatrix}
\end{align*}
最后,得出结果:
\[
	\begin{bmatrix}
	18 & 14\\
	62 & 66
	\end{bmatrix}
\]
\paragraph*{4.2-2}
为Strassen算法编写伪代码。
\paragraph*{解:}见代码。
\paragraph*{4.2-3}如何修改Strassen算法,使之适应规模不是2的幂的情况?证明:算法的运行时间为$\Theta(n^{lg7})$。
\paragraph*{解:}不断的将元素个数不是2的幂的矩阵进行拆分。这是一个思路,还未经过代码验证。
\end{document}
