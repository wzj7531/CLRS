\documentclass[a4paper,11pt]{article}
\usepackage{geometry}
\geometry{left=1.5cm,top=2cm,right=1.5cm}
\newcommand{\floor}[1]{\left\lfloor #1 \right\rfloor}
\newcommand{\ceiling}[1]{\lceil #1 \rceil}
\usepackage{clrscode3e}
\usepackage{ctex}
\usepackage{amsmath}
\usepackage{forest}
\usepackage{mathtools}
\begin{document}
\section*{4.4 用递归树求解递归式}
\paragraph*{}在递归树中,每个节点表示一个单一子问题的代价,子问题对应某次递归函数调用。我们将树中每层中的代价求和,得到每层代价,然后将所有层代价求和,得到所有层次的递归调用的总代价。以递归式$T(n)=3T(\floor{n/4})+\Theta(n^2)$为例,构造出下面的递归树:\\
{\footnotesize
\begin{forest}
	[$cn^2$
		[$c(\frac{n}{4})^2$ [$c(\frac{n}{16})^2$ [$\vdots$ [$T(1)$]] [$\vdots$] [$\vdots$ [$T(1)$]]] [$c(\frac{n}{16})^2$ [$\vdots$ [$T(1)$]] [$\vdots$] [$\vdots$ [$T(1)$]]] [$c(\frac{n}{16})^2$ [$\vdots$ [$T(1)$]] [$\vdots$] [$\vdots$ [$T(1)$]]]]
		[$c(\frac{n}{4})^2$ [$c(\frac{n}{16})^2$ [$\vdots$ [$T(1)$]] [$\vdots$] [$\vdots$ [$T(1)$]]] [$c(\frac{n}{16})^2$ [$\vdots$ [$T(1)$]] [$\vdots$] [$\vdots$ [$T(1)$]]] [$c(\frac{n}{16})^2$ [$\vdots$ [$T(1)$]] [$\vdots$] [$\vdots$ [$T(1)$]]]]
		[$c(\frac{n}{4})^2$ [$c(\frac{n}{16})^2$ [$\vdots$ [$T(1)$]] [$\vdots$] [$\vdots$ [$T(1)$]]] [$c(\frac{n}{16})^2$ [$\vdots$ [$T(1)$]] [$\vdots$] [$\vdots$ [$T(1)$]]] [$c(\frac{n}{16})^2$ [$\vdots$ [$T(1)$]] [$\vdots$][$\vdots$ [$T(1)$]]]]
	]
\end{forest}%
}
注意,舍入对求解递归式通常没有影响;其次,为了方便我们假设$n$是4的幂,这样所有子问题的规模均为整数,这就是我们构造递归树时需要忍受的两个不精确的地方。
\paragraph*{}因为子问题的规模每一步减少为上一步的$1/4$,又因为我们假设了$n$是4的幂,所以最终必然会达到边界条件。那么根节点与规模为1的子问题距离多远呢?深度为$i$的节点对应规模为$n/4^i$的子问题。因此,当$n/4^i=1$,或等价地$i=\log_{4}{n}$,子问题的规模变成1。因此,递归树有$\log_{4}{n}+1$层(深度为$0,1,2,\twodots,\log_{4}{n}$)。
\paragraph*{}接下来确定树的每一层代价。每层的节点数都是上一层的3倍,因此深度为$i$的结点数为$3^i$。因为每一层子问题的规模都是上一层的$1/4$,所以对$i=0,1,2,\towdots,\log_{4}{n}-1$,深度为$i$的每个结点的代价为$c(n/4^i)^2$。因此,深度为$i$的所有结点的总代价为$3^ic(n/4^i)^2=(3/16)^icn^2$。树的最底层深度为$\log_{4}{n}$,有$3^{\log_{4}{n}}=n^{\log_{4}{3}}$个结点,每个结点代价为$T(1)$,总代价为$n^{\log_{4}{3}}T(1)$,即$\Theta(n^{\log_{4}{3}})$,因为假定$T(1)$是常量。
\paragraph*{}现在求所有层次的代价之和,确定整棵树的代价:
\begin{equation}
	\begin{split}
		T(n)&=cn^2+\frac{3}{16}cn^2+(\frac{3}{16})^2cn^2+\towdots+(\frac{3}{16})^{\log_{4}{n}-1}cn^2+\Theta(n^{\log_{4}{3}})\\
		&=\sum_{i=0}^{\log_{4}{n}-1}cn^2+\Theta(n^{\log_{4}{3}})\\
		&=\frac{(3/16)^{\log_{4}{n}}-1}{(3/16)-1}+\Theta(n^{\log_{4}{3}})
	\end{split}
\end{equation}
最后的这个公式看起来比较凌乱,但我们可以再次充分利用一定程度的不精确,并利用无限递减几何级数作为上界。回退一步,我们得到:
\begin{equation}
	\begin{split}
		T(n)&=\sum_{i=0}^{\log_{4}{n}-1}cn^2+\Theta(n^{\log_{4}{3}})<\sum_{i=0}^{\infty}(\frac{3}{16})cn^2+\Theta(n^{\log_{4}{3}})\\
		&=\frac{1}{1-(3/16)}cn^2+\Theta(n^{\log_{4}{3}})\\
		&=\frac{16}{13}cn^2 + \Theta(n^{\log_{4}{3}})\\
		&=O(n^2)
	\end{split}
\end{equation}
\end{document}
