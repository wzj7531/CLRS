\documentclass[a4paper,11pt]{article}
\usepackage{geometry}
\geometry{left=2cm,top=2cm,right=2cm}
\newcommand{\floor}[1]{\left\lfloor #1 \right\rfloor}
\usepackage{clrscode3e}
\usepackage{ctex}
\usepackage{amsmath}
\usepackage{forest}
\title{算法基础}

\begin{document}

\section*{第 2 章\  算法基础}

\section*{2.1 插入排序}
	\begin{codebox}
		\Procname{$\proc{Insertion-Sort}(A)$}
		\li\For $j \gets 2$ \To $\attrib{A}{length}$
		\li	\Do
				$\id{key} \gets A[j]$
		\li		\Comment Insert $A[j]$ into the sorted sequence
					$A[1 \twodots j-1]$.
		\li		$i \gets j-1$
		\li		\While $i > 0$ and $A[i] > \id{key}$
		\li			\Do
						$A[i+1] \gets A[i]$
		\li				$i \gets i-1$					
					\End
		\li		$A[i+1]  \gets \id{key}$
			\End
	\end{codebox}

$T(n) = \Theta(n^2)$

\paragraph*{Ex2.1-2}
重写过程$\proc{Insertion-Sort}$,使之按非升序排序.
	\begin{codebox}
		\Procname{$\proc{Insertion-Sort}(A)$}
		\li\For $j \gets 2$ \To $\attrib{A}{length}$
		\li	\Do
		$\id{key} \gets A[j]$
		$A[1 \twodots j-1]$.
		\li		$i \gets j-1$
		\li		\While $i > 0$ and $A[i] < \id{key}$
		\li			\Do
		$A[i+1] \gets A[i]$
		\li				$i \gets i-1$					
		\End
		\li		$A[i+1]  \gets \id{key}$
		\End
	\end{codebox}

\paragraph*{Ex2.1-3}
线性查找问题.
	\begin{codebox}
		\Procname{$\proc{Linear-Find}(A,v)$}
		\li\For $j \gets 1$ \To $\attrib{A}{length}$
		\li	\Do
		\If	$A[j] \isequal v$
		\li		\Then \Return j	
				\End	
		\End
		\li	\Return \const{nil}
	\end{codebox}

\paragraph*{Ex2.1-4}
二进制数加法
	\begin{codebox}
		\Procname{$\proc{Binnary-Number-Addition}(A,B,n)$}
		\li	Let $C[1 \twodots n+1]$ be a new array.
		\li	$r \gets 0$
		\li\For	$i \gets n$ \Downto 1
		\li	\Do
		$C[i] \gets (A[i] + B[i] + r) \% 2$
		\li	$r \gets (A[i] + B[i] + r) / 2$
		\End
		\li$C[1] \gets r$
		\li \Return \const{C}
	\end{codebox}

\section*{2.2 分析算法}
\paragraph*{Ex2.2-2:选择排序}
首先找到A中最小元素并与A[1]进行交换,然后找出次最小元素并与A[2]进行交换...
\begin{codebox}
	\Procname{$\proc{Selection-Sort}(A)$}
	\zi\For 
	$i \gets 1$ \To $\attrib{A}{length}-1$
	\zi\Do
	$\id{min} \gets i$
	\zi\For 
	\Do 
	$j \gets i+1$ \To $\attrib{A}{length}$
	\zi \If $A[j] < A[\id{min}] $
	\zi \Then $min = j$
		\End
	\End
	\zi Exchange $A[min]$ and $A[i]$
	\End
\end{codebox}

\section*{2.3 设计算法}
\begin{codebox}
	\Procname{$\proc{Merge}(A, p, q, r)$}
	\li $n_1 \gets q - p + 1$
	\li $n_2 \gets r - q$
	\li let $L[1 \twodots n_1 + 1]$ and $R[1 \twodots n_2 + 1]$ be bew arrays
	\li \For $i \gets 1$ \To $n_1$
	\li \Do 
		$L[i] \gets A[p + i - 1]$
		\End
	\li \For $j \gets 1$ \To $n_2$
	\li \Do
		$R[j] \gets A[q + j]$
		\End
	\li $L[n_1 + 1] \gets \infty$
	\li $R[n_2 + 1] \gets \infty$
	\li $i \gets 1$
	\li $j \gets 1$
	\li \For $k \gets p$ \To $r$
			\Do
	\li			\If $L[i] \leq R[j]$
	\li				\Then 
						$A[k] \gets L[i]$
	\li					$i \gets i + 1$					
	\li				\Else
	\li 				$A[k] \gets R[j]$
	\li				  	$j \gets j + 1$
					\End
			\End
	
\end{codebox}

\begin{codebox}
	\Procname{$\proc{Merge-Sort}(A, p, r)$}
	\li \If $p < r$
	\li \Then
	$q \gets \floor{(p + r) / 2}$
	\li $\proc{Merge-Sort}(A, p, q)$
	\li $\proc{Merge-Sort}(A, q+1, r)$
	\li $\proc{Merge}(A, p, q, r)$
	\End
\end{codebox}
\noindent 利用递归树,可以得知$\proc{Merge-Sort}$的算法复杂度为$\Theta{(nlgn)}$

\paragraph*{Ex2.3-2}
重写过程$\proc{Merge-Sort}$,使之不使用哨兵,而是一旦数组L或R的所有元素均被复制回A就立刻停止,然后把另一个数组的剩余部分复制回A.
\begin{codebox}
	\Procname{$\proc{Merge-Sort-Without-Guard}(A, p, q, r)$}
	\li $n_1 \gets q -p + 1$
	\li $n_2 \gets r - q$
	\li Let $L[1 \twodots n_1]$ and $R[1 \twodots n_2]$ be new arrays.
	\li \For $i \gets 1$ \To $n_1$
	\li		\Do
				$L[i] \gets A[p + i - 1]$
			\End
	\li	\For $j \gets 1$ \To $n_2$
	\li 	\Do
				$R[j] \gets A[q + j]$
			\End
	\li $i \gets 1$
	\li	$j \gets 1$
	\li $k \gets p$
	\li \While $i \leq n_1$ and $j \leq n_2$
	\li		\Do
				\If $L[i] \leq R[j]$
	\li			\Then 
					$A[k] \gets L[i]$
	\li 			$i \gets i + 1$
	\li			\Else
	\li				$A[k] \gets R[j]$
	\li				$j \gets j + 1$
				\End
	\li			$k \gets k + 1$
			\End
	\li	\While $i \leq n_1$
	\li \Do
			$A[k] \gets L[i]$
	\li		$i \gets i + 1$
		\End
	\li \While $j \leq n_2$
	\li \Do
			$A[k] \gets R[j]$
	\li		$j \gets j + 1$
		\End
\end{codebox}
\paragraph*{Ex2.3-4}
我们可以把插入排序表示为如下的一个递归过程。为了排序$A[1 \twodots n]$,我们递归地排序$A[1 \twodots n-1]$,然后把$A[n]$插入已排序的数组$A[i \twodots n-1]$。为插入排序的这个版本的最坏情况运行时间写一个递归式。
\begin{codebox}
	\Procname{$\proc{Insertion}(A,p)$}
	\li $\id{key} \gets A[p]$
	\li $i \gets p-1$
	\li \While $i > 0$ and $A[i] < \id{key}$
	\li 	\Do
				$A[i + 1] = A[i]$
	\li			$i \gets i - 1$
			\End
	\li	$A[i + 1] \gets \id{key}$
\end{codebox}
\begin{codebox}
	\Procname{$\proc{Insertion-Sort-Recursive}(A, p)$}
	\li \If $p > 1$
	\li	\Then
			$\proc{Insertion-Sort-Recursive}(A, p-1)$
	\li		$\proc{Insertion}(A, p)$		
		\End
\end{codebox}
\noindent 其递归式如下:
\[
	T(n) = 
	\begin{cases}
		         \Theta(1) & \text{if $n = 1$}\\
		T(n-1) + \Theta(n) & \text{if $n = 2$}
	\end{cases}
\]
\paragraph*{Ex2.3-5}
假设A已排好序,为二分查找写出迭代式或递归的伪代码。证明二分查找的最坏情况运行时间为$\Theta(lgn)$
\begin{codebox}
	\Procname{$\proc{Binary-Search-Recursive}(A, key, low, high)$}
	\li \If $low \leq high$
	\li 	\Then
			$mid = \floor{(low + high) / 2}$
	\li		\If $\id{key} < A[mid]$
	\li			\Then \Return $\proc{Binary-Search-Recursive}(A, key, low, mid - 1)$	
	\li		\ElseIf $\id{key} > A[mid]$
	\li			\Then \Return $\proc{Binary-Search-Recursive}(A, key, mid + 1, high)$
	\li		\ElseNoIf
	\li			\Return $mid$
			\End
	\li	\Else
   	\li 	\Return \const{NIL}		
		\End	
\end{codebox}
\noindent 该算法的递归式为$T(n) \gets T(n/2) + \Theta(1)$,由递归树可知,其算法复杂度为:$\Theta(lg(n))$
\paragraph*{Ex2.3-6}
注意到过程$\proc{Insertion-Sort}$的第5-7行的while循环采用一种线性查找来(反向)扫描已排好序的子数组$A[1 \twodots j-1]$。我们可以使用二分查找来把插入排序的最坏情况总运行时间改进到$\Theta(nlgn)$吗?

\paragraph{}先将上面的$\proc{Binary-Search-Recursive}$算法修改为如下形式:
\begin{codebox}
	\Procname{$\proc{Binary-Search-Recursive}(A, key, low, high)$}
	\li \If $low \leq high$
	\li 	\Then
			$mid = \floor{(low + high) / 2}$
	\li		\If $\id{key} < A[mid]$
	\li			\Then \Return $\proc{Binary-Search-Recursive}(A, key, low, mid - 1)$	
	\li		\ElseIf $\id{key} > A[mid]$
	\li			\Then \Return $\proc{Binary-Search-Recursive}(A, key, mid + 1, high)$
	\li		\ElseNoIf
	\li			\Return $mid$
			\End
	\li	\Else
	\li 	\Return $\id{low}$	
		\End	
\end{codebox}
\begin{codebox}
	\Procname{$\proc{Insertion-Sort-Using-Binary-Search}(A)$}
	\li \For $j \gets 2 $ \To $\attrib{A}{length}$
	\li \Do
			$\id{key} \gets A[j]$
	\li		$i \gets j - 1$
	\li		$\id{index} \gets \proc{Binary-Search-Recursive}(A, key, 1, i)$
	\li 	\If $A[index] < \id{key}$
	\li			\Then 
				$\id{index} \gets \id{index} + 1$
				\End
	\li		\While $i \geq \id{index}$
	\li			\Do
				$A[i + 1] = A[i]$
	\li			$i = i - 1$
				\End
	\li		$A[i+1] = \id{key}$
		\End
	
\end{codebox}
\noindent 可见,在最坏情况下,就算使用了$\proc{Binary-Search-Recursive}$算法,也不能将其运行时间降低到$\Theta(nlgn)$。
\paragraph*{2.3-7}
描述一个运行时间为$/Theta(nlgn)$的算法,给定$n$个整数的集合$S$和另一个整数$x$,该算法能确定$S$中是否存在两个其和刚好为$x$的元素。

先用归并排序对S进行排序,其运行时间为$\Theta(nlgn)$,再使用二分查找算法查找$x-e$,其运行时间为$\Theta(nlgn)$,总运行时间为$\Theta(nlgn)$。

\section*{思考题}
\paragraph*{2-1}
在归并排序中对小数组采用插入排序,虽然归并排序在最坏情况运行下时间为$\Theta(nlgn)$,而插入排序的最坏情况运行时间为$\Theta(n^2)$,但是在插入排序中的常量银子可能使得它在$n$较小时,在许多机器上实际运行得更快。因此,在归并排序中当$n$较小时,采用插入排序来使递归的叶变粗是有意义的。考虑对归并排序的一种修改,其中使用插入排序来排序长度为$k$的$n/k$个子表,然后使用标准的合并机制来合并这些子序列,这里$k$是一个待定的值。

$a.$证明:插入排序最坏情况可以在$\Theta(nk)$时间内排序每个长度为$k$的$n/k$个子表。

根据插入排序的特点,排序长度为$k$的序列在最坏情况下的运行时间为$\Theta(k^2)$,那么排序$n/k$个这样的子序列则需要$\frac{n}{k}\Theta(k^2)=\Theta(nk)$。\\
$b.$表明在最坏情况下如何在$\Theta(nlg(\frac{n}{k}))$时间内合并这些子表。

首先画出其递归树:

\begin{forest}
[$cn$
	[$c\frac{n}{2}$ [$c\frac{n}{4}$ [$\vdots$ [$T(k)$]] [$\vdots$ [$T(k)$]]] [$c\frac{k}{4}$ [$\vdots$ [$T(k)$]] [$\vdots$ [$T(k)$]]]]
	[$c\frac{n}{2}$ [$c\frac{n}{4}$ [$\vdots$ [$T(k)$]] [$\vdots$ [$T(k)$]]] [$c\frac{k}{4}$ [$\vdots$ [$T(k)$]] [$\vdots$ [$T(k)$]]]]
]
\end{forest}\\
可知,该递归树总共有$lg(\frac{n}{k})+1$层,除了最后一层的代价为$\Theta(nk)$外,其他各层的代价都是$cn$,从最后一层向上都是在合并,所以总代价为:$cnlg(\frac{n}{k})=\Theta(nlg(\frac{n}{k}))$

\end{document}

