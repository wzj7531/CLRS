\documentclass[a4paper,11pt]{ctexbook}
\usepackage{mypreamble}
\begin{document}
\setcounter{chapter}{8}

\chapter{中位数和顺序统计量}
在一个由$n$个元素组成的集合中,第$ i $个\textbf{顺序统计量}(order statistic)是该集合中第$ i $小的元素。例如,在一个元素集合中,\textbf{最小值}是第1个顺序统计量$ (i = 1) $,\textbf{最大值}是第$ n $个顺序统计量$ (i=n)$。用非形式化的描述来说,一个\textbf{中位数}(median)是它所属集合的“中点元素”。当$ n $为奇数时,中位数是唯一的,位于$ i = (n+1)/2 $处。当$ m $为偶数时,存在两个中位数,分别位于$ i=n/2 $和$ i=n/2+1 $处。如果不考虑$ n $的奇偶性,中位数总是出现在$ i=\floor{(n+1)/2} $处(\textbf{下中位数})和$ i=\ceiling{(n+2)/2} $处(\textbf{上中位数})。为了简便起见,本书中所用的“中位数”都是指下中位数。

本章讨论\textbf{从一个由$ n $个互异的元素构成的集合中选择第$ i $个顺序统计量的问题}。为了方便起见,假设集合中的元素都是互异的,但实际上我们所作的都可以推广到集合中包含重复元素的情形。我们将这一问题形式化定义为如下的\textbf{选择问题}:

\textbf{输入:}一个包含$ n $个(互异的)数的集合$ A $和一个整数$ i $,$ 1\leq i \leq n $。

\textbf{输出:}元素$ x \in A $,且$ A $中恰好有$ i-1 $个其他元素小于它。

我们可以在$ O(nlgn) $时间内解决这个问题。因为我们可以用堆排序或归并排序对输入数据进行排序,然后在输出数组中根据下标找出第$ i $个元素即可。本章将介绍一些更快的算法。
\section{最小值和最大值}
在一个有$ n $个元素的集合中,需要多少次才能确定其最小元素呢?我们可以很容易地给出$ n-1 $次比较这个上界:依次变量集合中的每个元素,并记录下当前最小元素。在下面的程序中,我们假设该集合元素存放在数组$ A $中,且$ A.length=n $:
\begin{codebox}
	\Procname{$\proc{MINIMUM}(A)$}
	\li $ \id{min} \gets A[1] $
	\li \For $  \id{i} \gets 2 \To \attrib{A}{length} $
	\li  \Do
		  \If $ \id{min} > A[j] $
	\li    \Then 
		    $ \id{min} = A[i] $
		   \End
		 \End
   \li \Return $ \id{min} $
\end{codebox}

对于确定最小值问题,我们可以得到其下界就是$ n-1 $次比较,这就是我们能够得到的最好结果。

\textbf{同时找到最小值和最大值}

在$ n $个元素中同时找到最大值和最小值,只要分别独立地找出最大值和最小值,这各需要$ n-1 $次比较,共需要$ 2n-2 = \Theta(n)$次比较。事实上,我们只需要最多$ 3\floor{n/2} $次比较就可以同时找到最大值和最小值。具体的方法是记录已知的最小值和最大值。但并不是将每一个输入元素与当前的最小值和最大值进行比较——这样做的代价是每个元素都需要2次比较,而是对输入元素成对地进行处理。首先,我们将一对输入元素相互比较,然后把较小的与当前的最小值比较,把较大的与当前最大的比较。这样,对每两个元素共需要3次比较。

如何设定最小值和最大值依赖于$ n $的奇偶性。如$ n $是奇数,就将最小值和最大值设为第一个元素的值,然后成对的处理余下的元素。如果$ n $是偶数,就对前两个元素做一次比较,以决定最小值和最大值的初值,然后与$ n $是奇数的情形一样,成对的处理余下的元素。

现在分析一下总的比较次数。如果$ n $是奇数,那么总共需要$ 3\floor{n/2} $次比较。如果$ n $是偶数,则是先进行一次初始比较,然后进行
$ 3(n-2)/2 $次比较,共$ 3n/2-2 $次比较。因此无论哪种情形,总的比较次数至多是$ 3\floor{n/2} $。

\section*{练习}

\noindent 9.1-1 证明:在最坏情况下,找到$ n $个元素中第二小的元素需要$ n + \ceiling{lgn} - 2 $次比较。(提示:可以同时找最小元素。)

待解决

\noindent 9.1-2 证明:在最坏情况下,同时找到$ n $个元素中最大值和最小值的比较次数的下界是$ \ceiling{3n/2}-2 $。(提示:考虑有多少个数有成为最大值或最小值的潜在可能,然后分析一下每一次比较会如何影响这些计数。)

证明:当最小值和最大值都在最后一对元素中出现时,即为最坏情况。

当$ n $为奇数时,总共需要:
\begin{equation}
\begin{split}
3+\frac{3(n-3)}{2} &= 1 + 3\frac{(n-3)}{2} + 2\\
&=\frac{3n}{2}-\frac{3}{2}\\
&=(\ceiling{\frac{3n}{2}}-\frac{1}{2})-\frac{3}{2}\\
&=\ceiling{\frac{3n}{2}}-2
\end{split}
\end{equation}

当$ n $为偶数时,总共需要:
\begin{equation}
\begin{split}
1+\frac{3(n-2)}{2}&=\frac{3n}{2}-2\\
&=\ceiling{\frac{3n}{2}}-2
\end{split}
\end{equation}

\section{期望为线性时间的选择算法}

本节将介绍一种解决选择问题的分治算法。$ \proc{RANDOMIZED-SELECT} $算法是以第7章的快速排序算法为模型的。与快排一样,我们仍然将输入数组进行递归划分。但与快排不同的是,快排会递归处理划分的两边,而$ \proc{RANDOMIZED-SELECT} $只处理划分的一边。快排的期望运行时间是$ \Theta(nlgn) $,而$ \proc{RANDOMIZED-SELECT} $的期望运行时间为$ \Theta(n)$。这里假设输入数据都是互异的。

$ \proc{RANDOMIZED-SELECT} $利用了7.3节介绍的$ \proc{RANDOMIZED-PARTITION} $过程。与$\proc{RANDOMIZED-QUICKSORT}$一样,因为它的部分行为是由随机数生成器的输出决定的,所以$ \proc{RANDOMIZED-SELECT} $也是一个随机算法。以下是$ \proc{RANDOMIZED-SELECT} $的伪代码,它返回数组$ A[p \cdots r] $中第$ i $小的元素。

\begin{codebox}
	\Procname{$ \proc{RANDOMIZED-SELECT}(A,p,r,i)$}
	\li \If $ p \isequal r $
	\li \Then
			\Return $ A[p] $
		\End
	\li $ q \gets \proc{RANDOMIZED-PARTITION}(A,p,r) $
	\li $ k \gets q-p+1 $
	\li \If $ i \isequal k $
	\li	\Then \Return $ A[q] $
	\li	\ElseIf $ i < k $
	\li	\Then \Return $ \proc{RANDOMIZED-SELECT}(A,p,q-1,i)$
	\li \Else \Return $ \proc{RANDOMIZED-SELECT}(A,q+1,r,i-k) $		
		\End
	
\end{codebox}

$ \proc{RANDOMIZED-SELECT} $的最坏情况运行时间为$ \Theta(n^2) $,即使是找最小元素也是如此,因为在每次划分时可能极不走运地总是按余下的元素中最大的来进行划分,而划分操作需要$ \Theta(n)$时间。\textbf{我们也将看到该算法有线性的期望运行时间,又因为它是随机化的,所以不存在一个特定的会导致其最坏情况发生的输入数据。}

\end{document}
