\documentclass[a4paper,11pt]{article}
\usepackage{geometry}
\geometry{left=1.5cm,top=2cm,right=1.5cm}
\newcommand{\floor}[1]{\left\lfloor #1 \right\rfloor}
\newcommand{\ceiling}[1]{\lceil #1 \rceil}
\usepackage{clrscode3e}
\usepackage{ctex}
\usepackage{amsmath}
\usepackage{forest}
\usepackage{mathtools}

\begin{document}
\section*{4.5 用主方法求解递归式}
\subsection*{主定理}
\paragraph*{定理4.1(主定理)}令$a\geq 1$和$b>1$是常数,$f(n)$是一个函数,$T(n)$是定义在非负整数上的递归式:
\[
	T(n) = aT(n/b)+f(n)
\]
其中我们讲$n/b$解释为$\floor{n/b}$或$\ceiling{n/b}$。那么$T(n)$有如下的渐进界:
\begin{itemize}
	\item[1.]
		若对某个常数$\epsilon>0$有$f(n)=O(n^{log_b{a}-\epsilon})$,则$T(n)=\Theta(n^{log_b{a}})$。
	\item[2.]
		若$f(n)=\Theta(n^{log_b{a}})$,则$T(n)=\Theta(n^{log_b{a}}lgn)$。
	\item[3.]
		若对某个常数$\epsilon>0$有$f(n)=\Omega(n^{log_b{a}+\epsilon})$,且对某个常数$c<1$和所有足够大的$n$有$af(n/b)\leq cf(n)$,则$T(n)=\Theta(f(n))$
\end{itemize}		

\end{document}
