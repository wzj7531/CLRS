\documentclass[a4paper,11pt]{article}
\usepackage{geometry}
\geometry{left=2cm,top=2cm,right=2cm}
\newcommand{\floor}[1]{\left\lfloor #1 \right\rfloor}
\newcommand{\ceiling}[1]{\lceil #1 \rceil}
\usepackage{clrscode3e}
\usepackage{ctex}
\usepackage{amsmath}
\usepackage{forest}
\usepackage{mathtools}

\begin{document}
\section*{4.1 最大子数组问题}
\begin{codebox}
\Procname{$\proc{FIND-MAX-CROSSING-SUBARRAY}(A,low,mid,high)$}
\li $\id{left-sum} \gets -\infty$
\li $\id{sum} \gets 0$
\li \For $i \gets \id{mid} \Downto low$
\li 	\Do
			$\id{sum} \gets \id{sum} + A[i]$
\li		\If $\id{sum} > \id{left-sum}$
\li				\Then
					$\id{left-sum} \gets \id{sum}$
\li				$\id{max-left} \gets i$					
					\End
			\End
\li	$\id{right-sum} \gets -\infty$
\li $\id{sum} \gets 0$
\li \For $j \gets \id{mid} + 1 \To high$
\li		\Do
			$\id{sum} \gets \id{sum} + A[j]$
\li		\If $\id{sum} > \id{right-sum}$
\li			\Then
					$\id{right-sum} \gets \id{sum}$
\li				$\id{max-right} \gets j$	
				\End
			\End
			\li \Return $(\id{max-left},\id{max-right},\id{left-sum}+\id{right-sum})$
\end{codebox}

\begin{codebox}
	\Procname{$\proc{FIND-MAXIMUM-SUBARRAY}(A,low,high)$}
	\li	\If $\id{high} \isequal \id{low}$
	\li		\Then 
					\Return $(\id{low},\id{high},A[low])$
	\li		\Else 
					$\id{mid} \gets \floor{(\id{low}+\id{high})/2}$
	\li			$(\id{left-low},\id{left-high},\id{left-sum}) \gets \proc{FIND-MAXIMUM-SUBARRAY}(A,low,mid)$
	\li			$(\id{right-low},\id{right-high},\id{right-sum}) \gets \proc{FIND-MAXUMUM-SUBARRAY}(A,mid+1,high)$
	\li			$(\id{cross-low},\id{cross-high},\id{cross-sum}) \gets \proc{FIND-MAXIMUM-SUBARRAY}(A,low,mid,high)$
	\li			\If $\id{left-sum} \geq \id{right-sum}$ and $\id{left-sum} \geq \id{cross-sum}$
	\li				\Then \Return $(\id{left-low},\id{left-high},\id{left-sum})$
	\li				\ElseIf $\id{right-sum} \geq \id{left-sum}$ and $\id{right-sum} \geq \id{cross-sum}$
	\li				\Then \Return $(\id{right-low},\id{right-high},\id{right-sum})$
	\li				\ElseNoIf \Return $(\id{corss-low},\id{cross-high},\id{cross-sum})$
						\End
				\End
\end{codebox}
\section*{分治算法的分析}
首先,第1行花费常量时间,对于$n \gets 1$的基本情况,也很简单:第2行花费常量时间,因此,
\[
	T(1) = \Theta(1)
\]
当$n>1$时为递归情况。第1行和第3行花费常量时间,第4行和第5行求解的子问题均为$n/2$个元素的子数组(假定原问题规模为2的幂,保证了$n/2$为整数),因此每个子问题的求解时间为$T(n/2)$。因为需要求解两个子问题,因此第4行和第5行给总运行时间增加了$2T(n/2)$。又因为第6行花费$\Theta(n)$时间,第7-11行仅花费$\Theta(1)$时间。所以对于递归情况,我们有:
\[
	T(n)=
	\begin{cases}
		\Theta(1) & \text{若 $n=1$}\\
		2T(n/2) + \Theta(n) & \text{若$n>1$}
	\end{cases}
\]
上面递归式的解为$T(n)=\Theta(nlgn)$。
\section*{练习}
\paragraph*{4.1-1}
当A的所有元素均为负数时,$\proc{Find-Maximum-SubArray}$返回什么?\\
返回其中最大的那个负数及其在数组中的位置。
\paragraph*{4.1-2}
对最大子数组问题,编写暴力求解方法的伪代码,其运行时间应该为$\Theta(n^2)$
\begin{codebox}
	\Procname{$\proc{Find-Maximum-SubArray-Violence}(A)$}
	\li $\id{max-sum} \gets A[0]$
	\li $\id{left} \gets 0$
	\li $\id{right} \gets 0$
	\li $\id{temp-sum} \gets 0$
	\li \For $i \gets 1 \To \attrib{A}{length}-1$
	\li		\Do $\id{temp-sum}\gets A[i]$
	\li		\For $j \gets i+1 \To \attrib{A}{length}$
	\li			\Do $\id{temp-sum} = \id{temp-sum} + A[j]$
	\li			\If $\id{temp-sum} > \id{max-sum}$
	\li				\Then $\id{max-sum}=\id{temp-sum}$
	\li				$\id{left} = i$
	\li				$\id{right} = j$
						\End
					\End
				\End
	\li \Return $(\id{left},\id{right},\id{max-sum})$
\end{codebox}
\paragraph*{4.1-4}
假定修改最大子数组问题的定义,允许结果为空子数组,其和为0。你应该如何修改现有算法,使它们能允许空子数组为最终结果?\\
只需将$\proc{Find-Max-Crossing-SubArray}$修改如下(注意,这里假定数组下标从1开始):
\begin{codebox} 
  \Procname{$\proc{FIND-MAX-CROSSING-SUBARRAY}(A,low,mid,high)$}
  \li $\id{left-sum} \gets 0$
  \li $\id{sum} \gets 0$
  \li \For $i \gets \id{mid} \Downto low$
  \li   \Do
        $\id{sum} \gets \id{sum} + A[i]$ 
  \li   \If $\id{sum} > \id{left-sum}$
  \li       \Then 
            $\id{left-sum} \gets \id{sum}$
  \li       $\id{max-left} \gets i$
            \End
        \End
  \li $\id{right-sum} \gets 0$
  \li $\id{sum} \gets 0$
  \li \For $j \gets \id{mid} + 1 \To high$
  \li   \Do 
        $\id{sum} \gets \id{sum} + A[j]$
  \li   \If $\id{sum} > \id{right-sum}$
  \li     \Then
            $\id{right-sum} \gets \id{sum}$ 
  \li       $\id{max-right} \gets j$
          \End
        \End
  \li \Return $(\id{max-left},\id{max-right},\id{left-sum}+\id{right-sum})$
  \end{codebox}
\paragraph*{4.1-5}
使用如下思想为最大子数组问题设计一个非递归的、线性时间的算法。从数组的左边界开始,由左至右处理,记录到目前为止已经处理过的最大子数组。若已知$A[1 \dots j]$的最大子数组,基于如下性质讲解扩展为$A[1 \dots j+1]$的最大子数组:$A[1 \dots j+1]$的最大子数组要么是$A[1 \dots j]$的最大子数组,要么是某个子数组$A[i \dots j+1](1 \leq i \leq j+1)$。在已知$A[1 \dots j]$的最大子数组的情况下,可以在线性时间内找出形如$A[i \dots j+1]$的最大子数组。
\begin{codebox}
	\Procname{$\proc{Find-Maximum-Subarray-Linear}(A)$}
	\li	$\id{left} \gets 0$
	\li $\id{right} \gets 0$
	\li $\id{temp-sum} \gets A[0]$
	\li $\id{max-sum} \gets \id{temp-sum}$
	\li \For $j \gets 1 \To \attrib{A}{length}$
	\li		\Do $\id{temp-sum} \gets \id{temp-sum} + a[j]$
	\li		\If $\id{max-sum} < \id{temp-sum}$
	\li			\Then $\id{right \gets j}$
	\li			$\id{max-sum} \gets \id{temp-sum}$
	\li			\Else
	\li				\If $A[j] > \id{temp-sum}$
	\li					\Then $\id{left} \gets j$
	\li					$\id{right} \gets j$
	\li					$\id{temp-sum} \gets A[j]$
	\li					$\id{max-sum} \gets \id{temp-sum}$
							\End
					\End
				\End
	\li \Return $(\id{left},\id{right},\id{max-sum})$
\end{codebox}
\end{document}
